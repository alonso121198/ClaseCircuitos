\documentclass[a4paper,12pt]{article}
\usepackage[lmargin=1cm,rmargin=1cm,tmargin=1cm,bmargin=1.5cm]{geometry}
 % \usepackage[utf-8]{inputenc} (si lo pones tienes problemas y la verdad no se por que . 
\usepackage[spanish]{babel}%para que te aparezca capitulo 1 en vez de chapter 1
\usepackage{amsmath}
\usepackage{amsfonts}
\usepackage{amssymb}
\usepackage{float} % Este sive para usa el [H] en el begin figure

\usepackage{circuitikz} %este es el paquete para crear circuitos electricos 
\usepackage{siunitx} %este paquete es para las unidades de medida

\newtheorem{preg}{Pregunta} %este es para las preguntas 

\begin{document}

\begin{itemize}
\item Nombre : Juan Alonso Alcalá Lujan 
\item Codigo : 20172226A 
\end{itemize}

% PREGUNTA NUMERO 1 
\begin{preg}
Lea la guia, analice y realice los calculos que se indican en los pasos 2, 4 y 6.
\end{preg}

\begin{itemize}
\item \textbf{Paso 2} :

\begin{figure}[H]
\begin{center}
\begin{circuitikz} 
\draw (0,0) to[battery,a= \SI{10}{\volt}] (0,4)
             to[] (4,4) node[label={above:A}]{} to[R , l=$R_1$, *-*] (4,2) node[label={right:B}]{} to [R , l=$R_2$,*-*] (4,0) node[label={right:C}]{} ;
\draw (0,0) to [] (4,0) ;
\end{circuitikz} 
\end{center}
\caption{circuito1}
\end{figure}

	\begin{itemize}
	\item  Del circuito 1 seleccione $R_2$ tal que $\frac{V_{BC}}{V_{AC}} = \frac{1}{2}$ 
\begin{align*}
	V_{BC} &= R_2 I \\ 
	V_{AC} &= (R_1+R_2) I
\intertext{ Dividiendo tenemos }
	\dfrac{V_{AC}}{V_{BC}} &= \dfrac{R_1+R_2}{R_2} \\
	2 &=  \dfrac{R_1+R_2}{R_2} \\
	\therefore R_1 &= R_2  = \SI{10}{\kilo\ohm}  
\end{align*}

	\item Coloque una resistencia $R_3$ en paralelo con $R_2$, seleccione el valor de $R_3$ tal que $\frac{V_{BC}}{V_{AC}} = \frac{1}{4}$	
\begin{align*}
	V_{BC} &= \left(\dfrac{R_2 R_3}{R_2+R_3}\right) I \\
	V_{AC} &= \left(\dfrac{R_2 R_3}{R_2+R_3}+R_1\right) I 
\intertext{ Dividiendo tenemos }
	\dfrac{V_{BC}}{V_{AC}} &= \dfrac{\left(\dfrac{R_2 R_3}{R_2+R_3}\right) }{\left(\dfrac{R_2 R_3}{R_2+R_3}\right) +R_1} \\
	\dfrac{1}{4} &= \dfrac{\left(\dfrac{R_2 R_3}{R_2+R_3}\right) }{\left(\dfrac{R_2 R_3}{R_2+R_3}\right) +R_1} \\
\intertext{ como $R_1 = R_2$ tenemos }
\therefore & R_3 = \dfrac{R_2}{2} = \SI{5}{\kilo\ohm}
\end{align*}	
\end{itemize}	

\item \textbf{Paso4}

\begin{figure}[H]
\begin{center}
\begin{circuitikz} 
\draw (0,0) to[battery,a= \SI{10}{\volt}] (0,5)
             to[] (5,5) to[R , l=$R_2$]  (5,0) ;
\draw (0,0) to [] (5,0) ; 
\draw (3,5) to[R , l=$R_1$,a=\SI{10}{\kilo \ohm}] (3,0);
\end{circuitikz} 
\caption{Circuito2}
\end{center}
\end{figure}

\begin{itemize}
\item Calcule $R_2$ tal que $\frac{I_2}{I} = \frac{1}{2}$
\begin{align*}
V &= I_2 R_2 \\
V &= I \left(\dfrac{R_1 R_2}{R_1+R_2}\right) \\
\intertext{Igualando}
I_2 R_2 &= I \left(\dfrac{R_1 R_2}{R_1+R_2}\right) \\
\dfrac{I_2}{I} &= \dfrac{R_1 }{R_1+R_2} \\
 \dfrac{1}{2} &= \dfrac{R_1 }{R_1+R_2} \\
 \therefore  R_1 &= R_2 = \SI{5}{\kilo\ohm}
\end{align*}
\item Coloque una resistencia $R_3$ en serie con $R_2$, calcule el valor de $R_3$ tal que $\frac{I_2}{I} = \frac{1}{4}$
\begin{align*}
V &= (R_2 + R_3) I_2\\
V &= \left(\dfrac{R_1(R_2+R_3)}{R_1+R_2+R_3}\right)I \\
\intertext{Igualando y separando tenemos }
\dfrac{1}{4} &=  \dfrac{R_1}{R_1+R_2+R_3} \\
\therefore R_3 &=  \dfrac{R_1}{2} = \SI{5}{\kilo\ohm}
\end{align*}
\end{itemize}

\end{itemize}

	















\begin{preg}
¿Que significa que un multimetro digital sea de $3\frac{1}{2}$ digitos ? 
\end{preg}
Un multimetro digital de $ 3 \frac{1}{2} $ muestra 3 digitos completos en pantalla que pueden cambiar de 0 a 9 y un digito que puede ser bien 0 o 1 . Significa que en la pantalla se mostraran numeros desde 0 a 1999 .
$$
\underbrace{1}_{1/2} \underbrace{\quad 3 \quad 4 \quad 1 }_{3 \mbox{ digitos} }
$$
Lo de arriba es lo que se mostrara en pantalla . 
\end{document}
