\documentclass[a4paper,12pt]{article}
\usepackage[lmargin=1cm,rmargin=1cm,tmargin=1cm,bmargin=1.5cm]{geometry}
 % \usepackage[utf-8]{inputenc} (si lo pones tienes problemas y la verdad no se por que . 
\usepackage[spanish]{babel}%para que te aparezca capitulo 1 en vez de chapter 1
\usepackage{amsmath}
\usepackage{amsfonts}
\usepackage{amssymb}
\usepackage{float} % Este sive para usa el [H] en el begin figure
\parindent=0mm %con esto elimino la sangría 
\usepackage{graphicx}
\usepackage{caption} % Necesario para poner caption
\usepackage{subcaption} % Necesario para las subfiguras 



\usepackage{circuitikz} %este es el paquete para crear circuitos electricos 
\usepackage{siunitx} %este paquete es para las unidades de medida

\newtheorem{preg}{Pregunta} %este es para las preguntas 


\begin{document}

\section*{Anexo- Hoja de datos}

\begin{itemize}

\item \textbf{Participantes:} 
\begin{itemize}
\item Juan Alonso Alcalá Lujan 20172226A
\item Lopez Lozano Christian Rafael 20180016B
\end{itemize} 
\item \textbf{Fecha:} 09-06-2020

\end{itemize}


\subsection*{Paso1}

\begin{itemize}

\item \textbf{Osciloscopio} 
\begin{itemize}

\item Marca y modelo : Tektronix , TDS 2024
\item Ancho de banda :  200MHz 
\item Señal de calibracion : 5V/1kHz

\end{itemize}


\item \textbf{Generador} 

\begin{itemize}
\item marca y modelo : - - - - - - - - - -
\item numero y digitos : - - - - - - - - -

\end{itemize}

\end{itemize}



\subsection*{Paso2}


\begin{itemize}
\item \textbf{voltaje pico}: 5 voltios 
\item \textbf{frecuencia} : 1 kHz 
\end{itemize}

\subsection*{Paso3}

\begin{center}
 \begin{tabular}{|c|| c c c c|} 
 \hline
 frecuencia nominal & \SI{100}{\hertz} & \SI{1}{\kilo\hertz} & \SI{10}{\kilo\hertz} & \SI{1}{\mega\hertz}  \\  
 \hline\hline
 perido medido & 10 ms & 1ms & 0.1 ms & 1 us \\
 \hline
 frecuencia calculada & 100 Hz & 1kHz & 10kHz & 1MHz \\
 \hline
\end{tabular}
\end{center}

\subsection*{Paso4\footnote{los errores son los mismos para todas las medidas de frecuencia}}

\begin{center}
 \begin{tabular}{|c|| c c c c|} 
 \hline
 frecuencia nominal & \SI{100}{\hertz} & \SI{1}{\kilo\hertz} & \SI{10}{\kilo\hertz} & \SI{1}{\mega\hertz}  \\  
 \hline\hline
$V_{rms}$(medido) & 3.535V & 3.535V  & 3.535V  & 3.535V  \\ 
 \hline
$V_{rms}$(calculado)& 3.536V & 3.536V & 3.536V & 3.536V \\
 \hline 
  $\% \varDelta V_{rms}/V_{rm} $ & 0.028 \% & 0.028 \% & 0.028 \% & 0.028 \%\\
 \hline
\end{tabular}
\end{center}


\newpage

\subsection*{Paso5}


\begin{center}
 \begin{tabular}{|c| c | c | c | c|} 
 \hline
 frecuencia nominal & $V_p$ & Periodo & $ \tau = RC $  \\  
 \hline\hline
 100 Hz & $5 V$ & 10 ms & \SI{1}{\mega\hertz}  \\   
 \hline
 1000 Hz  & $5 V$ & 1000 us & \SI{1}{\mega\hertz}  \\  
 \hline 
 2000 Hz & $5 V$ & 500 us & \SI{1}{\mega\hertz}  \\  
 \hline
  4	000 Hz & $5 V$ & 250 us & \SI{1}{\mega\hertz}  \\  
  \hline
\end{tabular}
\end{center}


\begin{figure}[H]
     \centering
     \begin{subfigure}[b]{0.3\textwidth}
         \centering
         \includegraphics[width=\textwidth]{f=100Hz.pdf}
         \caption{100 Hz}
     \end{subfigure}
     \begin{subfigure}[b]{0.3\textwidth}
         \centering
         \includegraphics[width=\textwidth]{f=1000Hz.pdf}
         \caption{1000 Hz}
     \end{subfigure}
          \centering
     \begin{subfigure}[b]{0.3\textwidth}
         \centering
         \includegraphics[width=\textwidth]{f=2000Hz.pdf}
         \caption{100 Hz}
     \end{subfigure}
     \hfill
          \centering
     \begin{subfigure}[b]{0.3\textwidth}
         \centering
         \includegraphics[width=\textwidth]{f=4000Hz.pdf}
         \caption{100 Hz}
     \end{subfigure}
\end{figure}







% Significado de la constante : Equivale al tiempo necesario para que el condensador se cargue o descarge un 63 \% de la carga final o inicial respectivamente . 

	
\subsection*{Paso7\footnote{Preguntar sobre la diferencia entre $\phi$ calculado y $\phi$ medido}}


\begin{center}
 \begin{tabular}{|c| c | c | c | c | c |} 
 \hline
 $C_1$ & $f$ & $\phi_{cal}$ & $ V_{ep} $ & $V_{sp}$ & $ \phi_{med} $ \\  
 \hline\hline
 200 nF & 500 Hz & 57.87 & 5 V & 4 V & 32.4  \\   
 \hline
 330 nF & 500 Hz & 43.98 &  5 V &  3.4 V & 46.8 \\  
 \hline 
\end{tabular}
\end{center}

\begin{figure}[H]
     \centering
     \begin{subfigure}[b]{0.3\textwidth}
         \centering
         \includegraphics[width=\textwidth]{f=500Hz, C1=200nF.pdf}
         \caption{f=500Hz, C1=200nF	}
     \end{subfigure}
     \begin{subfigure}[b]{0.3\textwidth}
         \centering
         \includegraphics[width=\textwidth]{f=500Hz, C2=330nF.pdf}
         \caption{f=500Hz, C2=330nF}
     \end{subfigure}
\end{figure}

\subsection*{Paso8\footnote{No puedo poner el error debido a esa cuestion entre $\phi$ calculado y $\phi$ medido  }}

\begin{center}
 \begin{tabular}{|c| c | c | c | c | c |} 
 \hline
 $C_1$ & $f$ & a & b & $\phi$  	& $\% \dfrac{\Delta \phi}{\phi} $ \\  
 \hline\hline
 200 nF & 500 Hz & 4.5 & 8.5 & 31.96 &  \\   
 \hline
 330 nF & 500 Hz & 10 &  14 & 45.58 &  \\  
 \hline 
\end{tabular}
\end{center}	


\begin{figure}[H]
     \centering
     \begin{subfigure}[b]{0.3\textwidth}
         \centering
         \includegraphics[width=\textwidth]{Grafica_f=500_C=200nF.pdf}
         \caption{Grafica f=500,C=200nF}
     \end{subfigure}
     \begin{subfigure}[b]{0.3\textwidth}
         \centering
         \includegraphics[width=\textwidth]{Grafica_f=500Hz, C1=3300nF.pdf}
         \caption{Grafica f=500Hz, C1=3300nF}
     \end{subfigure}
\end{figure}


  

\end{document}